\documentclass[11pt]{article}

\usepackage[utf8]{inputenc}
\usepackage[T1]{fontenc}
\usepackage{pxfonts}

\newcommand{\showcase}{0123456789 ABCDEFGHIJKLMNOPQRSTUVWXYZ k}

\title{Blackboard letters from DejaVu fonts for \TeX}
\author{Rogério Theodoro de Brito}
\date{2011-10-09}

%% This works for plain TeX
\font\blackboardfont=fdjr8r at 10pt

%% This works for LaTeX
\newfont{\bbold}{fdjr8r at 10pt}


\begin{document}
\maketitle
\section{Introduction}

It is a frequent question among users of \TeX/\LaTeX{} on how one can use
blackboard fonts, similar to those provided by the AMS fonts with the
command \texttt{mathbb}, but with a wider variety of symbols. There are many
possibilities, like the use of the \emph{fourier} package (with its
\texttt{fourierbb} font), various double-strock fonts, and the \emph{bbold}
package.

The DejaVu fonts created by the DejaVu project has widely extended the work
of Bitstream in their Vera family of fonts and one of such extensions is the
inclusion of characters beyond the Basic Multiplane of Unicode with the
glyphs of blackboard fonts. The production of the DejaVu project is very
aesthetically pleasing and more complete than some of the offerings of
today.

It is possible to argue that with newer engines like Xe\TeX{} and Lua\TeX,
and efforts like Will Robertson's \emph{fontspec} and \emph{unicode-math},
and rich OpenType fonts (like Khaled Hosny's \emph{XITS} fonts), such
problems are completely solved for new texts.

But there are many users that don't want to (or can't) change their set up
for many reasons (not the least for writing texts with coauthors).

With this in mind, this very tiny subset o the DejaVu Sans fonts, made ready
for use with regular \TeX, \LaTeX, and pdf\TeX{} allows the users some wider
range of choice in the typesetting of their current texts.

\section{Examples}

For \TeX: {\blackboardfont \showcase}

Regular: \showcase

For \LaTeX: {\bbold \showcase}
\end{document}
